\documentclass[12pt, twoside]{scrartcl}
\usepackage{physics}
\usepackage{amsmath}
\usepackage[english]{babel}
\usepackage[math-rm]{siunitx}
\usepackage{amssymb}
\usepackage{wasysym}
\usepackage{mathtools}

%Seitenlayout
\usepackage{geometry}
\geometry{
a4paper,
top=25mm,
inner=30mm,
outer=25mm,
bottom=25mm,
headheight=12mm,
footskip=12mm
}

\usepackage[headsepline]{scrpage2}
\clearscrheadfoot
\pagestyle{scrheadings}
\ihead{Thimo Preis}
\ohead{\pagemark}

\setlength{\parindent}{0pt}

%Referenzieren und Hyperlinks
\usepackage[english]{varioref}

\usepackage{hyperref}

\usepackage[english, noabbrev]{cleveref}


\begin{document}
\section{Central question}
MAYBE ALSO PUT PICTURE OF MEASURED POWER SPECTRUM IN FOR REFERENCE ?
Welcome to the talk about my Bachelor thesis concerned with a \emph{first order perturbative treatment of the cosmic density-fluctuation power spectrum in the Zel'dovich approximation}.
We will begin with an introduction to the central question of this thesis and continue by introducing the underlying theoretical framework of Kinetic Field Theory.
The explicit evaluation of our goal takes place in the third section, after which we conclude this talk with a summary and a potential outlook.
\\
\\
Cosmic structure formation is quantified by the density power spectrum, which describes the density contrast of the universe as a function of scale. On large scales, gravity competes with cosmic expansion, and structures grow according to linear theory. Considering this linear regime of cosmic structure formation, the evolution behaviour of the density power spectrum is obtained from standard perturbation theory, which can be reproduced by Kinetic Field Theory already in the free theory.
Whether the linear growth behaviour is also obtained upon transitioning into the interacting regime of KFT was the central question of my thesis.
We therefore essentially want to check, whether the first order perturbative corrections next to the free power spectrum vanish in the linear regime.
\\
\\
Let us therefore introduce a corresponding formulation of KFT, in order to explicitly evaluate the density power spectrum to first order in the perturbations relative to Zel'dovich trajectories which will model the underlying particle dynamics.
\section{Kinetic Field Theory}
\subsection{KFT in a nutshell}
Kinetic Field Theory is a microscopic, non-equilibrium, statistical field theory for initially
correlated ensembles of classical particles obeying Hamiltonian dynamics.
The full microscopic phase-space dynamics of a gravitationally interacting ensemble consisting
of individual particles are encoded in a central generating functional.
The initial probability distribution on the phase space is mapped to any later
time by means of the Hamiltonian flow.
\subsection{Zel'dovich approximation}
We employ the Zel'dovich approximation for the underlying dynamics of our statistical system consisting of classical point particles moving on an expanding spatial background.
It is an approximative description of Hamiltonian
dynamics which assumes particles to move on inertial trajectories.
The essential step of the Zel’dovich approximation is now to extrapolate these trajectories to the present day.
\\
 \\
The corresponding equations of motion are then found to be given by the following. (refer to eq. (5)).
The first term in the momentum equation of motion accounts for the gravitational interaction between the particles, whereas the second term can be considered as an external force field $\vec{F}_d$ linearly dependent on the particle momentum, which we will call \emph{drag force}.
\\
\\
We choose the perturbations as such that the "free" motion is provided by absence of the drag force and the gravitational interaction. This
choice is motivated by the expectation that the gravitational interaction will be partially
compensated by the drag force, such that the resulting contribution due to particle interactions would
become small.\\
Treating the corresponding contribution in the action as an inhomogeneity will then lead to a perturbation 
theory relative to pure Zel’dovich trajectories. \\
\subsection{The Zel'dovich approximation within Kinetic Field Theory}
Having arrived at a concise expression for the canonical generating functional of KFT,
one can proceed to solve for statistical quantities in a perturbative manner, where a Taylor expansion of the full generating
functional in terms of the interaction operator $\hat{S}_I$ to first order will be our course of action.
\\
\\
We find the following perturbative corrections to the density power spectrum by expanding the interaction operator to first order and calculating the two-point density cumulant, where the former term is sourced by the gravitational interaction and the latter term is sourced by the presence of a drag force. We now want to assess whether these corrections indeed vanish in the linear regime, such that the linear growth behaviour is consistently reproduced by KFT upon transitioning into the interacting regime. An isolated analysis of the respective corrections appearing here in terms of their contribution to the free density power spectrum in the linear regime is what the following section is dedicated to.
\section{The density power spectrum to first order in the perturbations}
\subsection{The Drag cumulant part one}
The perturbative correction sourced by the drag force decomposes into two non-vanishing contributions, where the first contribution is displayed here in terms of powers in the initial power spectrum. Note that
the blue line indicates the free power spectrum and that the black curve indicates the contribution of the corresponding perturbative correction.
\\ \\
The first contribution encodes a considerable enhancement of structures on small scales, especially by the terms non-linear in the initial power spectrum (which is represented by the orange curve). This accumulation of structures in the small scale regime is due to the presence of the drag force itself, which we can understand further if we turn towards the second contribution.
\subsection{The Drag cumulant part two}
This figure shows the second contribution to the perturbative correction sourced by the drag force, where dashed lines imply negative values.
The linear contribution to the drag cumulant (indicated by red) substantially suppresses structure formation on large
and intermediate scales, since the exerted drag force prevents confluent particle streams of forming larger structures.
The presence of the drag force altogether increases the abundance of structures on small
scales by slowing down particle trajectories, which amounts to suppressing structure formation on large
and intermediate scales, therefore transporting power from large and intermediate to small scales (\emph{emphasise by going back to part one}).
\subsection{Comparison of drag and interaction cumulant}
Considering the perturbative correction sourced by the gravitational interaction of particles, we observe
that the gravitational attraction encoded in them results in
confluent particle streams, such that a considerable amount of power is generated on large and
intermediate scales (which is indicated by the blue curve).\\
\\
Considering their joint contribution (which is represented by the black curve), the perturbative corrections sourced by the drag force and the gravitational interaction cancel effectively exactly in the intermediate to large scale regime (which is represented by black curve on large scales).
Beyond-free structure growth encoded in the cumulants describing the effects of gravitational interaction is compensated due to the drag exerted on confluent streams in the large scale regime. 
We can therefore conclude that the linear growth of the density power spectrum, as obtained in standard perturbation theory and already reproduced by KFT in the free theory, is preserved upon taking perturbations to first order into account.
\subsection{The full first order perturbed density power spectrum}
The full density power spectrum to first order in the perturbations is therefore obtained as an accumulation of all the results we have discussed so far.
We observe how the suppression of structure growth beyond small scales due to the deceleration of confluent streams by the drag force (which is indicated by the orange curve) and the generation of structures due to gravitational interaction generating said confluent streams in the first place (here represented by the red curve) are counteracting each other on the respective scales.
The abundance of structures on small scales is substantially enhanced due to the nature of the Zel’dovich approximation describing free inertial motion of particles and due to the drag force slowing down confluent streams. Qualitatively, this is the behaviour found for the non-linear $\Lambda$CDM density power spectrum.\\
\\
Our identification of perturbative contributions relative to the free motion of particles rests on the
assumption that the resulting contribution to the action would be small with respect to
the free action. From the compensation effect derived for the additional contributions emerging to first order in the perturbations relative to the free theory we can conclude that our framework is a physically feasible approach in the linear regime. Considering the non-linear regime however, the employed
Zel’dovich approximation itself prevents us from accurately reproducing results from
numerical N-body simulations. The free motion of particles along Zel’dovich trajectories leads to the dispersion of structures after stream crossing due to their free streaming motion. The reason behind this is that in the Zel'dovich approximation when different streams cross they pass each other without interacting (show with arms crossing and flying through),
because the evolution of fluid elements, which we call particles, is local \footnote{Note that the evolution of fluid elements at this order is lo-
	cal, i.e. it does not depend on the behavior of the rest of fluid elements}. As a result, high-density regions become washed
out. This is represented in our result, since the abundance of high-density structures is way to small compared to results from measurements or N-body simulations. 
Only considering the gravitational interactions and the drag force to first order, we are
approximately describing the full Hamiltonian dynamics of the particles and are thus
only partially compensating the dispersion of structures.
This plot is the final result of this thesis in that it encapsulates the complete physical
information contained in the framework of KFT by reproducing the linear behaviour as
predicted by standard perturbation theory and the qualitative non-linear behaviour of
structure growth as it is measured for the non-linear ΛCDM density power spectrum.
\section{Conclusion}
To conclude, we were able to show that the results of standard perturbation theory are reproduced to first order in the interactions in the linear regime, by using a formulation of KFT which accounts for the Zel’dovich approximation including a drag force.\\
\\
The enhancement of power in the non-linear regime of the density power spectrum, as
observed in numerical N-body simulations of the ΛCDM non-linear power spectrum, is
qualitatively reproduced by the accumulation of structures due to the presence of a drag
force.\\
\\
It remains to be seen how well the compensation of the perturbative corrections in
the linear regime and the enhancement of structure formation in the non-linear regime
translates to higher orders in the perturbations.
\\
\\
This concludes my talk and I want to thank you all for listening
\newpage.
\section{Questions}

\begin{description}
	\item[Linear and Non-Linear] But also take into consideration:\\
	Non-linear evolution causes density-perturbation modes to couple:
	while modes of different wave lengths evolve independently during
	linear evolution, mode coupling in the non-linear evolution moves
	power from large to small scales as structures collapse. The effect
	on the power spectrum is that the amplitude on small scales is
	increased at the expense of intermediate scales. Large scales
	continue to evolve linearly and independently.\\
	Therefore, large scales are also the linear regime in the sense that they are the only ones growing linearly if one takes the whole theory into consideration. \\
	\\
	Even if the original density perturbation field $\delta$ is Gaussian, it must
	develop non-Gaussianities during non-linear evolution. This is
	evident because $\delta \geq -1$ by definition, but can become arbitrarily
	large. An originally Gaussian distribution of δ thus becomes
	increasingly skewed as it develops a tail towards infinite δ.\\
	\\
	As long as the growth remains linear, structures do
	not change their size against the expanding background.
	A Fourier analysis in wave numbers co-moving with the
	mean cosmic expansion shows that the Fourier modes of
	the density-contrast field neither couple to each other nor
	change their wave number with time. Large-scale density-
	fluctuation modes are still linear today.\\
	\\
	Also consider that: As the density contrast approaches unity, its evolution
	becomes nonlinear. In the course of nonlinear evolution,
	overdensities contract, causing matter to flow from
	larger to smaller scales. Power in the density-fluctuation
	field is thus transported toward smaller modes or toward
	larger wave numbers k. This mode coupling process de-
	forms the power spectrum on small scales, i.e., for large
	k.!
	\item[Growth of Perturbations during radiation-dominated era] Why does radiation pressure suppress structures to grow ? What about dark matter, which does not interact electromagnetically ?
	$\Rightarrow$ This has to be treated relativistically. One derives the linearised evolution equations for scalar perturbations in Newtonian gauge, which yield evolution
equations of the Newtonian potential. One finds From the conservation of the stress-tensor, we derived the relativistic generalisations of
the continuity equation and the Euler equation. Also, a Poisson equation and an equation for the curvature perturbations is obtained. Any mode of interest for observations today was outside the Hubble radius if we go back sufficiently far into the past. Inflation sets the initial condition for these superhorizon modes.
On superhorizon scales, this equation yields that the newtonian gravitational potential is constant, such that (via Poisson) the density contrast is constant.
On superhorizon scales, the density perturbations are therefore simply proportional to the curvature
perturbation set up by inflation.
Below that in the radiation dominated era, from the same set of equations it also follows that subhorizon modes of the newtonian gravitaitonal potential oscillate with a frequency and an amplitude that decays as $a^{−2}$.
As pre-
dicted, the potential is constant when the modes are outside the horizon. Two of the modes enter
the horizon during the radiation era. While they are inside the horizon during the radiation
era their amplitudes decrease as a −2 . The resulting amplitudes in the matter era are therefore
strongly suppressed. During the matter era the potential is constant on all scales. The longest
wavelength mode in the figure enters the horizon during the matter era, so its amplitude is only suppressed by the factor of 9/10 coming from the radiation-to-matter transition.During the
matter era, the subhorizon fluctuations in the radiation density therefore oscillate with constant
amplitude around a shifted equilibrium point\\
\\
\\
If a mode enters the horizon during the radiation-
dominated era, its growth will cease. Instead, the modes will oscillate due to the
radiation pressure. To be more precise: the radiation-baryon “fluid” will oscillate.
What happens to the CDM will be discussed below. These oscillations of the radiation-
baryon fluid are called baryonic acoustic oscillations,
Once the radiation decouples from the baryons (at$z \approx 1100$) the radiation no longer
behaves like a fluid, and the oscillations cease. In fact, already slightly before that,
when matter starts dominating the energy density, the radiation pressure will become
less effective at preventing the growth of modes.
The stalling of growth and the formation of oscillations is a property of the radiation-
baryon fluid. Of more interest to us is, however, the behavior of the cold dark matter
(CDM), because dark matter halos will later be the birthplace of galaxies. During the
radiation-dominated era it is the radiation fluid that produces the growth of modes.
The CDM only interacts with the radiation fluid through gravity NOW SUPERHORIZON. And this interaction
only goes one way: The CDM reacts to the gravitational potential of the radiation, but
not vice versa, because radiation dominates the mass, since we have nearly one billion photons per baryons in our Universe. The density of the CDM can
increase at some point simply because CDM starts streaming into the gravitational
well produced by a perturbation in the radiation fluid. SO THIS IS SUBHORIZON: If a mode enters the horizon during the radiation-dominated era and the radiative fluid
starts oscillating, this does not necessarily mean that the CDM behaves in the same
way. In fact, on small enough scales the effect of the fluctuation in the gravitational po-
tential on the CDM can be regarded as averaged out. The CDM perturbations can now
only grow through their own gravity, which is much weaker than the radiation-driven
perturbations through which the CDM perturbations grew before. ESSENTIALLY, CDM DOESNT GROW BECAUSE IT WOULD ONY GROW IN THE GRAVITATIONAL WELLS OF PERTURBATIONS IN THE RADIATION-BARYON FLUID (THIS ONLY WORKS IN THIS DIRECTION SINCE WE HAVE NEARLY A BILLION PHOTONS PER BARYON ?? DOES THE LATTER REASON APPLY FOR CDM ?). THEN,  SUBHORIZON BARYONIC-ACOUSTIC OSCILLATIONS OF THE RADIATION BARYON FLUID PREVENT THE CDM TO FLOW INTO THE GRAVITATIONAL WELLS OF RADIATION PERTURBATIONS. THEREFORE, CDM CAN ONLY GROW BY ITS SELF-GRAVITY, WHICH IS MUCH WEAKER THAN THE RADIATION DRIVEN PERTURBATIONS THROUGH WHICH THE CDM PERTURBATIONS GREW BEFORE. [-> Also this:The main constituents of the cosmic fluid were dark
matter, baryons, electrons, and photons. Overdensities
in the dark matter compressed the fluid due to their
gravity until the rising pressure in the tightly coupled
baryon-electron-photon fluid was able to counteract
gravity and drive the fluctuations apart. In due course,
the pressure sank, gravity won again, and so forth: the
baryon-electron-photon fluid underwent acoustic oscilla-
tions but not the dark matter, which was decoupled.] DECOUPLED IN THE SENSE OF (I THINK) CDM HAS NO JEANS LENGTH SINCE DARK MATTER BEHAVES LIKE AN ENSEMBLE OF COLLISION-LESS PARTICLES WITH VANISHING VELOCITY DISPERSION. ?? This effectively
stalls the growth of the CDM perturbations.T his stalling of the growth has a very strong consequence for the power spectrum
of CDM density perturbations at the time of the CMB decoupling as well as for the
power spectrum of the CMB anisotropies we see on the sky. Basically one expects for
all k which enter the horizon after a eq (the scale factor a at the time of matter-radiation
equilibrium) to have a power spectrum similar in shape to the initial one ($P(k) \propto k$),
but grown by a factor of $(a_{eq} /a_{end, infl} )^4$.\\
\\
Also: The main constituents of the cosmic fluid were dark
matter, baryons, electrons, and photons. Overdensities
in the dark matter compressed the fluid due to their
gravity until the rising pressure in the tightly coupled
baryon-electron-photon fluid was able to counteract
gravity and drive the fluctuations apart. In due course,
the pressure sank, gravity won again, and so forth: the
baryon-electron-photon fluid underwent acoustic oscilla-
tions but not the dark matter, which was decoupled.
Since the pressure was dominated by the photons, whose
pressure is a third of their energy density, a good ap-
proximation to the sound speed of the tightly coupled
photon-baryon fluid was ..Only such density fluctuations could undergo acoustic
oscillations which were small enough to be crossed by
sound waves in the available time.\\
\\

	\item[HZP power spectrum of CDM at early time in matter-dominated era] The Harrison-Zel'dovich-Peebles spectrum inferred at the end of inflation with $P \propto k$ as it provides (nearly) scale invariant superhorizon scalar and tensor perturbations. This comes about since one can show that The potential
	fluctuation caused by the perturbation remain constant (due to slow roll condition of the inflaton field) during
	inflation, such that inflation predicts approximately identical
	potential fluctuations on all accessible physical scales.
	\\
	Bartelmann argument: Recall that its [of the HZP power spectrum] shape was inferred from the
	simple assumption that the mass of density fluctuations entering
	the horizon should be independent of the time when they enter the
	horizon. For the power spectrum of cold dark matter-
	density fluctuations in the Universe, Recall that its shape was inferred from the
	simple assumption that the rms mass of density fluctua-
	tions entering the horizon should be independent of the
	time when they enter the horizon and from the fact that
	perturbation modes entering during the radiation era
	are suppressed until matter begins dominating.
	THESE ARE EQUIVALENT, INITIALLY IT WAS DERIVED FROM THIS ASSUMPTION OF TIME INDEPENDENCE, BUT NOWADAYS IT NATURALLY COMES ABOUT FROM SLOW ROLLING INFLATON FIELD.
	\\
	\\
	The modes grow \begin{itemize}
		\item $\propto a^2$ outside of the horizon in the radiation-dominated era
		\item Not at all inside the horizon during radiation domination
		\item $\propto a$ in the matter-dominated era inside the horizon
	\end{itemize}
	\item[Statistical properties of density contrast, power spectrum and autocorrelation function] 	A random field is called statistically homogeneous if all the joint multipoint probability distribution functions $ p(1,2,...)$ or its moments, ensemble averages of local density products, remain
	the same under translation of the coordinates $x_1,x_2,...$ in space (here $i=(x_i)$). Thus the probabilities depend only on the relative positions. A stochastic field is called statistically isotropic if
	$p(1,2,...) $is invariant under spatial rotations. We will assume that cosmic fields are statistically
	homogeneous and isotropic, as predicted by most cosmological theories.\\
	The two-point correlation function is defined as the joint ensemble average of the density at two
	different locations,
	\begin{equation*}
	\xi(r) = <\delta(\vec{x}) \delta(\vec{x}+\vec{r})>
	\end{equation*}
	which depends only on the norm of r due to statistical homogeneity and isotropy. The power spectrum is a well-de9ned quantity for almost all homogeneous random fields. This
	concept becomes, however, extremely fruitful when one considers a Gaussian field. It means that
	any joint distribution of local densities is Gaussian distributed. Any ensemble average of product of
	variables can then be obtained by product of ensemble averages of pairs. We write explicitly this => Wick theorem for density field (odd number of fields average vanish, whereas even number separates in product of averages over pairs.)  This is why the Bispectrum for Gaussian distribution vanishes, but the trispectrum would not ! The statistical properties of the random variables $\delta(\vec{k})$ are then entirely determined by the shape and normalization of$ P(k)$.
	In most cases $<\delta>= 0$ and the above equations simplify considerably. In the following we usually
	denote $\sigma^2=<\delta^2> - <\delta>^2$ as the local second-order cumulant. The Wick theorem then implies that in case of a Gaussian
	field $\sigma^2$ is the only non-vanishing cumulant. \\
	\\The power spectrum solely depends on the absolute value of the wave vectors due to isotropy.
	The density contrast is a random field, which must be isotropic
	and homogeneous in order to comply with the fundamental cos-
	mological assumptions. This means that the statistical properties
	of $\delta$, e.g. its mean or variance, do not change under rotations and
	translations.
	The mean of the density contrast vanishes per definition.
	The variance of the density contrast defines the power spectrum, where $\delta_D$ is Dirac’s delta distribution, which ensures that modes of
	different wave vector $\vec{k}$ are uncorrelated in Fourier space in order
	to ensure homogeneity. The power spectrum cannot depend on the
	direction of $\vec{k}$ because of isotropy.\\
	\\
	What is the power spectrum and how can it be measured ?\\
	The definition (1.22) shows that the power spectrum is given by
	an average over the Fourier modes of the density contrast. This
	average extends over all Fourier modes with a wave number k,
	i.e. it is an average over all directions in Fourier space keeping
	k constant. In other words, Fourier modes are averaged within
	spherical shells of radius k.In configuration space, structures can be quantified by the (two-
	point) correlation function
	\[\xi(x) = <\delta(\vec{x} \delta(\vec{x}+\vec{y}))>\]
	where the average is now taken over all positions ~y and all ori-
	entations of the separation vector $\vec{x}$ , assuming homogeneity and
	isotropy. The correlation function ξ is the Fourier transform of
	the power spectrum. Assuming isotropy implies:
	\[ \xi(x) = \int \frac{d^3 k}{(2\pi)^3} P(k) e^{-i \vec{k}\cdot \vec{x}} =  1/(2 \pi^2) \int_0^{\infty} dk\; k^2 P(k) \frac{\sin(kx)}{kx}\]
	The power spectrum can therefore be inferred as
	
	\[P(k) = 4 \pi \int dx \; x^2 \xi(x) \frac{\sin{kx}}{kx}\]
	where the correlation function can be measured by counting galaxy pairs and use the correlation function of the galaxies as an estimate for the underlying matter distribution Further page 69 on observing big bang to shot noise and direct measurement of power spectrum.  However one understands the averaging process in the correlation function $\xi$ as an average over a volume or an average over a large number of volumes with $\vec{x}$ fixed inside them. The homogeneity and isotropy of the Universe implies that these two definitions lead
	to the same answer. Mathematically put: the perturbation is ergodic.\\
	\\
	The density contrast in the Universe
	is a Gaussian random field means that the probability for finding a
	value between $\delta$ and $\delta + d\delta$ is given by a Gaussian distribution;\\
	\\
	Statistics in Cosmology: initial fluctuations are generated by inflation, amplification by self-gravity, statistical description necessary, density field
	\begin{equation}
	\delta = \frac{\rho -\bar{\rho}}{\bar{\rho}} \quad \mathrm{with} \quad \bar{\rho} = \Omega_m \rho_{crit}.
	\end{equation}
	Distribution of values of $\delta \rightarrow$ Gaussian random field, specify distirbution function and correlations:
	\[	P(\delta(\vec{x}), \delta(\vec{y})) = \frac{1}{\sqrt{2 \pi \det(C)}} \exp{-1/2 
		\begin{pmatrix}
		\delta(\vec{x})\\
		\delta(\vec{y})
		\end{pmatrix}^T			
		C^{-1}
		\begin{pmatrix}
		\delta(\vec{x})\\
		\delta(\vec{y})
		\end{pmatrix}}
	\]
	with covariance matrix
	\[ C =  \begin{pmatrix}
	\delta^2(\vec{x}) & \delta(\vec{x}) \delta(\vec{y})\\
	\delta(\vec{y}) \delta(\vec{x}) & \delta^2(\vec{y})\\
	\end{pmatrix}\].
	The correlation function is then defined via
	\[\xi({x}) = <  \delta(\vec{x})\delta(\vec{y}+\vec{x})>\],
	if $\xi = 0$, then the Gaussian separates $P(\delta(\vec{x}),\delta(\vec{y}))= p(\delta(\vec{x})) p(\delta(\vec{y}))$ and the two field values are statistically independent. As $\vec{x}\rightarrow\vec{y}$, the correlation function becomes the variance $\xi\rightarrow <\delta^2(\vec{x})>$. For  homogeneous field:
	statistical properties only depend on the separation vector $\vec{r}=\vec{x}-\vec{y}$ and not on the location $\vec{x}$, then 
	\[\xi(\vec{r}=0) = <\delta(\vec{x})^2> = <\delta(\vec{y})^2>\]
	and $\xi(\vec{x},\vec{y})= \xi(\vec{r})$.\\
	The spectrum then describes the variance of Fourier modes
	\begin{align*}
	\delta(\vec{k}) &= \int d^3 x \delta(\vec{x}) e^{- i \vec{k}\vec{x}}\\
	\delta(\vec{x})&= \int \frac{d^3k}{(2 \pi)^3} \delta(\vec{k}) e^{i \vec{k} \vec{x}}
	\end{align*} 
	Now assume homogeneous fields, i.e.
	\[d^3 y = d^3 r \quad \mathrm{at\; fixed} \quad \vec{q}, \quad  \xi(\vec{x},\vec{y})=<\delta( \vec{x}) \delta(\vec{y})>.\]
	Then, 
	\begin{align*}
	<\delta(\vec{k}), \delta(\vec{k_1})^* > &= \int d^3x \int d^3y \; <\delta(\vec{y})\delta(\vec{x})> e^{-i \vec{k}\vec{x}+ i \vec{k_1}\vec{y}}  \\
	&= \int d^3r \xi(\vec{r}) e^{-i \vec{k}_1\vec{r}} \int d^3x e^{-i (\vec{k}-\vec{k_1})\vec{x}} \\
	&= P(k) \quad \quad (2 \pi)^3 \delta_D(\vec{k}-\vec{k_1}).
	\end{align*}
	with the power spectrum as the Fourier transform of the correlation function. If the random field is also isotropic, $P(k)$ and $\xi(\vec{r})$ depend only on scale or separation, not on the direction.\\
	\\
	The power spectrum has unit $Mpc^3$ since $P(k) \times \delta_D(\vec{k})$ is unit less due to it defining the variance of the density contrast which itself is unit less
	\\
	\\
	The density distribution is usually smoothed with a filter W R of a given size, R, commonly a
	top-hat or a Gaussian window. Indeed, this is required by the discrete nature of galaxy catalogs
	and N -body experiments used to simulate them.
	\\
	\\If the cosmic fields are Gaussian, their power spectrum (or, equivalently, their two-point correlation function) completely describes the statistical properties.
	However, as we saw in Section 2, the dynamics of gravitational instability is non-linear, and therefore
	non-linear evolution inevitably leads to the development of non-Gaussian features
	\\
	\\
	In this review we are concerned about time evolution of the cosmic 9elds during the matter
	domination epoch. In this case, as we discussed in Section 2, di=usion e=ects are negligible and
	the evolution can be cast in terms of perfect Juid equations that describe conservation of mass and
	momentum. In this case, the evolution of the density 9eld is given by a simple time-dependent
	scaling of the “linear” power spectrum
	
	\begin{equation*}
	P(k,t) = D_1^2(t) P_L(k),
	\end{equation*}
	where $D_1$ is the growing part of the linear growth factor. One must note, however, that the “lin-
	ear” power spectrum speci9ed by $P_L(k)$ derives from the linear evolution of density fluctuations
	through the radiation domination era and the resulting decoupling of matter from radiation. This evolution must be followed by using general relativistic Boltzmann numerical codes. The end
	result is that
	\[P_L(k)=k^{n_s} T^2 (k)\]
	where $n_s$ is the primordial spectral index ($n_s$ = 1 denotes the canonical scale-invariant spectrum), $T(k)$ is the transfer function that describes the evolution of the density field perturbations through decoupling ($T (0) ≡ 1$).
	
	
	
	\item[ How are the potential, velocity and density power spectrum related ?]
	The potential and the density are related via the Poisson equation
	\[
	k^2 \delta\Phi(k) \propto - \delta(k)\], whereas the density is similarly related linearly to the velocity potential via continuity
	\begin{equation}
		(\nabla)^2 \psi = - \delta^i
	\end{equation} MORE PRECISE ? The pressure perturbations are related to the density perturbations via equation of state.
	Everything is linearly connected, such that non-Gaussianities in the density power spectrum are similarly translated to the velocity and potential power spectrum.
	Thus, given
	the velocity potential, both density contrast and peculiar
	velocity will be determined. The velocity potential also
	has to be a Gaussian random field.
	\item[Jeans Length, dispersion less CDM] The sound speed defines the Jeans length, below which perturbations cannot grow, but oscillate. For dark matter consisting of
	weakly interacting massive particles, for instance, the concept of a
	sound speed makes no sense because the dark matter behaves like
	an ensemble of collision-less particles.forms of dark matter with vanishing velocity dispersion $v \rightarrow 0$ are called “cold
	dark matter” (CDM). They have $\lambda_J \rightarrow 0$, hence structures can
	grow on all scales.
	\[			\lambda_J \propto \frac{1}{a \sqrt{<v^{-2}>}} \sqrt{\pi/G/\rho_0 } \].
	Then the initial power spectrum with $ns=1$ is the shape of the spectrum for cold dark matter (CDM).
	For hot dark matter (HDM), it is cut off above the Jeans wave
	number k J corresponding to the finite velocity dispersion of the
	hot particles.
	\item Something on Zel'dovich approximation and forbidden spherical collapse due to eigenvalues of deformation tensor ?
	\item[Expansion and collision time scale]  At early times, curvature and cosmological constant are negligible,
	thus Friedmann’s equation implies
	\[\dot{a} = a \sqrt{8 \pi G \rho/3} \quad \Rightarrow \quad t_{exp} \approx \frac{a}{\dot{a}} = \sqrt{\frac{3}{8 \pi G \rho}}\].
	During radiation-dominated era $\rho \propto a^{-4}$ and during matter-dominated era $\rho \propto a^{-3}$, such that
	\[ t_{exp, rad} \propto a^{2}, \qquad \qquad t_{exp,matter} \propto a^{3/2}  \]

\item[time line]  $z=10^{10}$ Big Bang nucleosynthesis, $z=10^4$ matter-radiation equality, $z=10^3$ CMB release/formation of atoms.
\item  Maybe the factor $(2 \pi)^{-3}$ I had to put into my equations per hand stems from parsevals theorem and me calculating in the wrong space since
\[ 1/(2\pi)^3 \sigma^2_{\delta}.(\vec{k}) = \sigma^2_{\delta}(\vec{x})\]
\item[Structure formation standard model]
The current explanation of the large-scale
structure of the universe is that the present distribution of matter on cosmological scales results from
the growth of primordial, small, seed fluctuations on an otherwise homogeneous universe ampli9ed
by gravitational instability. Tests of cosmological theories which characterize these primordial seeds
are not deterministic in nature but rather statistical, for the following reasons. First, we do not have
direct observational access to primordial fluctuations (which would provide de9nite initial conditions
for the deterministic evolution equations). In addition, the time scale for cosmological evolution is
so much longer than that over which we can make observations, that it is not possible to follow the
evolution of single systems. In other words, what we observe through our past light cone is di=erent
objects at different times of their evolution; therefore, testing the evolution of structure must be done
statistically.
The observable universe is thus modeled as a stochastic realization of a statistical ensemble of
possibilities. The goal is to make statistical predictions, which in turn depend on the statistical
properties of the primordial perturbations leading to the formation of large-scale structures. Among
the two classes of models that have emerged to explain the large-scale structure of the universe, the
physical origin of stochasticity can be quite di=erent and thus give rise to very di=erent predictions.
The most widely considered models, based on the inJationary paradigm [279], generically give
birth to adiabatic 10 Gaussian initial fluctuations, at least in the simplest single-9eld models [602,304,
280,20]. In this case the origin of stochasticity lies on quantum fluctuations generated in the early
universe; all fluctuations originate from scalar adiabatic perturbations of the inflaton field

\item[On the Zel'dovich approximation] 
 So far we have dealt with density and velocity 9elds and their equations of motion. However,
it is possible to develop non-linear PT in a di=erent framework, the so-called Lagrangian scheme,
by following the trajectories of particles or Juid elements [705,102,465], rather than studying the 
dynamics of density and velocity fields. 5 In Lagrangian PT, the object of interest is the diplacement field $(q)$ which maps the initial particle positions q into the final Eulerian particle
positions x,
\[q = q^i + \psi(q^i,t)\]
One can derive an evolution equation for the displacement field $\psi$, where the extrapolation of its linear solution to the present day is given by the Zel'dovich approximation:
Note that the evolution of fluid elements at this order is
local, i.e. it does not depend on the behavior of the rest of fluid elements. The Zel’dovich approximation (hereafter ZA) [705] consists in using the linear displacement 9eld
as an approximate solution for the dynamical equations. The ZA is a local approximation and becomes
the exact dynamics in one-dimensional collapse.A significant shortcoming of the ZA is the fact that after shell crossing (“pancake formation”),
matter continues to flow throughout the pancake without ever turning around, washing out structures
at small scales. This can be fixed phenomenologically by adding some small e=effective viscosity to
Eq. (105), which then becomes the Burgers’ equation. This is the so-called adhesion approximation.\\
\\ 
The ZA [705] is one of the rare cases in which exact (non-perturbative) results can be obtained.
However, given the drastic approximation to the dynamics, these exact results for the evolution
of clustering statistics are of limited interest due to their restricted regime of validity. The reason
behind this is that in the ZA when di=erent streams cross they pass each other without interacting,
because the evolution of fluid elements is local. As a result, high-density regions become washed
out. Nonetheless, the ZA often provides useful insights into non-linear behavior.
For Gaussian initial conditions, the full non-linear power spectrum in the ZA can be obtained
as follows. Changing
from Eulerian to Lagrangian coordinates, the Fourier
transform of the density 9eld is
\begin{equation*}
\delta(\vec{k}) = \int d^3q^i e^{i\vec{k}(\vec{q}^i+\vec{\psi}(\vec{q}^i)))} .
\end{equation*}
With $\psi$ being the displacement field. The power spectrum therefore is
\begin{equation*}
P(k) = \int d^3q^i e^{i \vec{k}\cdot \vec{q}^i}<e^{i \vec{k} \cdot \Delta\psi(\vec{q}^i)}>,
\end{equation*}
where $\Delta\psi(\vec{q}^i)=\psi(\vec{q}^i_1)-\psi(\vec{q}^i_2)$, and $\vec{q}^i = \vec{q}^i_1-\vec{q}^i_2$. For Gaussian initial conditions the ZA displacement
is a Gaussian random 9eld, so Eq. (180) can be evaluated in terms of the two-point correlator of the displacement field.
\\
\\
Since the inertial
motion of particles on Zel’dovich-type trajectories already
contains part of the gravitational interaction, even a first-
order calculation in the microscopic perturbation theory
captures the nonlinear evolution of the power spectrum
over a wide range of scales remarkably well,\\
\\
 An analysis of the trajectories of classical, gravitating
point particles in an expanding (Friedman-Lemaître-
Robertson-Walker) space-time quickly leads to the con-
clusion that the comoving displacement of particles from
their initial positions is finite in reality even in the limit
of infinite times. In sharp contrast, the remarkably suc-
cessful Zel’dovich approximation [1] asserts that particle
trajectories are well approximated by trajectories which
resemble inertial motion in a suitable time coordinate,
which manifestly leads to unbounded displacements. Beginning with the retarded Green’s functions of
classical point particles in a static and in an expanding
space-time, and suitably regrouping the terms in the point-
particle Hamiltonian in an expanding space-time, we derive
the effective gravitational potential acting on point particles
relative to free Zel’dovich trajectories. The result shows
that, while the density field is evolving linearly, this
effective potential acts only for a short period of time at
early cosmic times. This is a direct consequence of the
relation between the initial density contrast and the initial
particle velocities enforced by the matter continuity equa-
tion. This result may contribute to clarifying the astounding
success of the Zel’dovich approximation.\\
\\
$\tau=D_+-1$ transformation of time coordinate: We do this because
the Zel’dovich approximation simply corresponds to free
inertial motion in this time coordinate:
\begin{equation*}
\vec{q}(\tau) = \vec{q}^{(i)} + \tau \vec{p}^{(i)}
\end{equation*}Particles can only travel a finite distance even over infite time (since the particle propagator
is bounded from above) even in an infinite time. This behavior can intuitively be
understood: Relative to the expanding space-time, free
particles slow down in comoving coordinates because their
initial momentum falls behind the cosmic expansion.
The effective gravitational potential experienced by
particles moving on Zel’dovich trajectories acts at
early cosmic times and for a short period of time
only, until nonlinear evolution sets in much later. At
the initial time, this effective gravitational potential
vanishes exactly because the initial particle velocity
is constrained by the continuity equation for the
density contrast. This contributes to explaining why
the Zel’dovich approximation is so good even
though its particle trajectories differ grossly from
those expected from the Hamiltonian for point
particles in an expanding space-time: The inertial
motion in the Zel’dovich approximation captures a
substantial fraction of the gravitational interaction
that would otherwise be necessary to accelerate the
bound Hamiltonian trajectories of free particles. The
gravitational acceleration relative to the Zel’dovich
trajectories is much weaker than it needs to be
relative to the bound trajectories.
However, the Zel’dovich approximation overshoots
substantially at late times.
\\
\\
The correct Greens solution to the rescaled equations of motion in my choice of time coordinates is given by 
\[ g_{qp}(t,t_i) = \int_{t_i}^{t} d\hat{t} \frac{1}{g(\hat{t})}\]
where $g(t)$ is exactly simply the time dependent function we used to rescale Lagrangian. Then one replaces $g_{qp}$ by Zel'dovich propagator to simplify everything. This is exactly done in KFT Review, but without rescaling momentum variable !. Note: 
It is important to note that $ḡ q p (t , t 0 )$ is limited from above
because of the cosmic expansion. Consequently, the iner-
tial trajectories described by this Green’s function deviate
strongly from the true, fully interacting trajectories. It will
thus be advantageous to find a replacement for $ḡ q p$ that
already captures part of the gravitational interaction. An
example for such a replacement is given by the Zel’dovich
approximation [46], but we prefer a slightly more general
approach here.\\
A frequent and convenient choice
in cosmology is the Zel’dovich approximation [46], which
describes particle trajectories as inertial in the time coor-
dinate $t = D_+$ 
\begin{equation*}
	g_{qp}(t,t_0)=t-t_0
\end{equation*}
Inserting this particular choice into (56) and comparing
to (50) immediately results in effective force to be 
\[ f(t) = - \frac{\dot{g}}{g} \dot{q} - \nabla \phi \]\\
\\

\item[Zel'dovich inside KFT]
 For late-time cosmologi-
cal applications using the unbound Zel’dovich propagator (my propagator), the correlation operator $\mathcal{C}(p)\approx1$ safely.\\
\\
The drag amplitude decreases with wavenumber, hence it has the strongest effect on the largest structures => Confluent streams/large structures are predominantly affected.\\
\\
Since
we assume that the initial density fluctuations are a sta-
tistically isotropic and homogeneous random field, the
momentum correlations can only depend on the absolute
value of the relative particle separation.With the Gaussian initial distribution (67) of particle po-
sitions and momenta, the momenta can immediately be
integrated out in the free generating functional. With the momentum-correlation matrix depending only
on the relative particle separations, this remaining expression can be fully factorised\\
\\
\item[Different things]
\item By
the Helmholtz theorem, the peculiar velocity field can be
decomposed into a curl and a gradient. By the cosmic ex-
pansion and angular-momentum conservation, the curl
component will quickly decay, so we can model the initial
peculiar velocity as the gradient of a velocity potential Ψ.
\item read a bit about grand canonical ensemble transition in felix phd thesis if björn should ask ..
\item Was das Auftauchen von mehr Wechselwirkungskumulanten als Dragkumulanten bei höheren Ordnungen angeht: Man braucht ja immer eine gleiche Ordnung von beiden zum Wegheben. In zweiter Ordnung wird es drei non-mode-coupling Terme aus der Wechselwirkung geben. In der Dragkumulante zweiter Ordnung gibt es zwei Ableitungsoperatoren, die am Ende auf drei verschiedene Weisen zu Termen der Art des dritten Terms in der ersten Ordnung führen. Es könnte also durchaus sein, dass die Cancelung auch bei höheren Ordnungen noch klappt.

\item Felix calls interaction cumulant non-mode-coupling term
\item The fast expansion due to the radiation density inhibited
further growth of such structures until the matter den-
sity started dominating. ??
\item Notice that Saha’s equation contains the inverse of the
$\eta$ parameter. which is a huge number due to
the high photon-to-baryon ratio in the Universe. This
counteracts the exponential which would otherwise
guarantee that recombination happens when kT  $\approx E_{ion,H}$ , i.e.,
at T ⬇ 1.6⫻ 10 5 K. Recombination is thus delayed by the
high photon number, which illustrates that newly formed
hydrogen atoms are effectively reionized by sufficiently energetic photons until the temperature has dropped
well below the ionization energy.

\item The matter power spectrum describes the density contrast of the universe (the difference between the local density and the mean density) as a function of scale. It is the Fourier transform of the matter correlation function. On large scales, gravity competes with cosmic expansion, and structures grow according to linear theory. In this regime, the density contrast field is Gaussian, Fourier modes evolve independently, and the power spectrum is sufficient to completely describe the density field. On small scales, gravitational collapse is non-linear, and can only be computed accurately using N-body simulations. Higher-order statistics are necessary to describe the full field at small scales.

\item[The Story of structure formation so far, important bits]

The early universe was dominated by radiation; in this case density fluctuations larger than the cosmic horizon grow proportional to the scale factor, as the gravitational potential fluctuations remain constant. Structures smaller than the horizon remained essentially frozen due to radiation domination impeding growth. As the universe expanded, the density of radiation drops faster than matter (due to redshifting of photon energy); this led to a crossover called matter-radiation equality at ~ 50,000 years after the Big Bang. After this all dark matter ripples could grow freely, forming seeds into which the baryons could later fall. The size of the universe at this epoch forms a turnover in the matter power spectrum which can be measured in large redshift surveys.(Point of $k_{max}$ in HZP spectrum can be understood as the turning point of matter radiation equality, there growth also had a turnover.)
\\
\\
The theory of what happened after the universe's first 400,000 years is one of hierarchical structure formation: the smaller gravitationally bound structures such as matter peaks containing the first stars and stellar clusters formed first, and these subsequently merged with gas and dark matter to form galaxies, followed by groups, clusters and superclusters of galaxies.\\
Difficult are only the first 400000 years...\\
\\
hese perturbations are thought to have a very specific character: they form a Gaussian random field whose covariance function is diagonal and nearly scale-invariant. Observed fluctuations appear to have exactly this form, and in addition the spectral index measured by WMAP—the spectral index measures the deviation from a scale-invariant (or Harrison-Zel'dovich) spectrum—is very nearly the value predicted by the simplest and most robust models of inflation. Another important property of the primordial perturbations, that they are adiabatic (or isentropic between the various kinds of matter that compose the universe), is predicted by cosmic inflation and has been confirmed by observations.
\\
\\
The primordial plasma would have had very slight overdensities of matter, thought to have derived from the enlargement of quantum fluctuations during inflation. Whatever the source, these overdensities gravitationally attract matter. But the intense heat of the near constant photon-matter interactions of this epoch rather forcefully seeks thermal equilibrium, which creates a large amount of outward pressure. These counteracting forces of gravity and pressure create oscillations, analogous to sound waves created in air by pressure differences.

These perturbations are important, as they are responsible for the subtle physics that result in the cosmic microwave background anisotropy. In this epoch, the amplitude of perturbations that enter the horizon oscillate sinusoidally, with dense regions becoming more rarefied and then becoming dense again, with a frequency which is related to the size of the perturbation. If the perturbation oscillates an integral or half-integral number of times between coming into the horizon and recombination, it appears as an acoustic peak of the cosmic microwave background anisotropy. (A half-oscillation, in which a dense region becomes a rarefied region or vice versa, appears as a peak because the anisotropy is displayed as a power spectrum, so underdensities contribute to the power just as much as overdensities.
\\
\\
Dark matter plays a crucial role in structure formation because it feels only the force of gravity: the gravitational Jeans instability which allows compact structures to form is not opposed by any force, such as radiation pressure. As a result, dark matter begins to collapse into a complex network of dark matter halos well before ordinary matter, which is impeded by pressure forces. Without dark matter, the epoch of galaxy formation would occur substantially later in the universe than is observed.
\\
\\
The physics of structure formation in this epoch is particularly simple, as dark matter perturbations with different wavelengths evolve independently. As the Hubble radius grows in the expanding universe, it encompasses larger and larger disturbances. During matter domination, all causal dark matter perturbations grow through gravitational clustering. However, the shorter-wavelength perturbations that are included during radiation domination have their growth retarded until matter domination. At this stage, luminous, baryonic matter is expected to mirror the evolution of the dark matter simply, and their distributions should closely trace one another.

It is a simple matter to calculate this "linear power spectrum"\\
\\
When the perturbations have grown sufficiently, a small region might become substantially denser than the mean density of the universe. At this point, the physics involved becomes substantially more complicated. When the deviations from homogeneity are small, the dark matter may be treated as a pressureless fluid and evolves by very simple equations. In regions which are significantly denser than the background, the full Newtonian theory of gravity must be included.\\
\\

\end{description}

\end{document}